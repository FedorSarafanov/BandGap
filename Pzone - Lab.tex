\input{text/pre}

\renewcommand{\phi}{\varphi}

\begin{document}

\def\labauthors{Виноградов И.Д., Шиков А.П.}
\def\labgroup{430}
\def\labnumber{1}
\def\labtheme{Измерение ширины запрещенной зоны}
\input{text/titlepage}

% \tableofcontents

\newpage
\section*{Введение}
Ждем Илюшину выборку по теории


\newpage
\section{Методика измерений}

\begin{figure}[h!]
	\centering
	\includegraphics[width = .9\linewidth]{img/scheme.jpg}
	\caption{Электрическая схема для измерений удельной электропроводности методом компенсации}
	\label{fig:5.1}
\end{figure}

Регулируемый источник тока (1) задаёт ток образца $I_\text{об}$, измеряемый амперметром $A1$. Регулируемый источник тока
(2) задаёт ток компенсации $I_\text{к}$ через эталонный резистор $R_\text{э}$, величина этого тока измеряется
амперметром $А2$. Напряжение $U_{ab}$ между зондовыми электродами a и b сравнивается с напряжением компенсации $U_k$ на
эталонном резисторе $R_\text{э}$ при помощи индикатора компенсации V.

При проведении измерений нужно установить ток образца, затем, изменяя ток компенсации, добиться нулевых показаний
индикатора компенсации V. В этом случае напряжение $U_k$ на эталонном резисторе $R_\text{э}$ будет равно напряжению $U_{ab}$:
\begin{equation}
	U_{ab}=U_{k}=I_{k} R_{\text{э}} 
	\label{eq:5.1}
\end{equation}

В реальной ситуации между зондовыми электродами будут паразитные потенциалы, связанные, во-первых, с влиянием
переходного сопротивления на контактах «образец – подводящие провода», во-вторых, появлением термоЭДС на контактах
полупроводника с металлом при нагреве образца. Для того чтобы устранить влияние этих потенциалов, измерение тока
компенсации производится дважды. Получив первый отсчёт $I_{k1}$, изменяем направление тока через образец и через
эталонный резистор, опять добиваемся равенства напряжений $U_k$ и $U_{ab}$, снимаем второй отсчёт $I_{k2}$. Обратите
внимание, что полярность разности потенциалов между электродами a и b, вызванная протеканием тока через образец, как и
напряжение на $R_\text{э}$, сменились на противоположные, а паразитные потенциалы, зависящие от свойств контактов, и
термоЭДС, зависящая от температуры образца, остались прежние. Таким образом, среднеарифметическое значение 

$$I_k=\frac{I_{k1}+I_{k2}}{2}$$

будет содержать информацию только о полезной составляющей напряжения $U_{ab}$.

Величину падения напряжения $U_k$ легко подсчитать:
$$U_{k}=I_{k} R_{\text{э}}$$

Величину сопротивления участка образца расположенного между зондовыми электродами a и b ($R_{\text{об}}$) можно определить из равенства:
$$R_{\text{об}}=\frac{U_{k}}{I_{\text{об}}}=\frac{I_{k} R_{\text{э}}}{I_{\text{об}}}$$
Зная размеры образца: a - ширина (см), d - толщина (см), l - расстояние между электродами a и b (см), можно рассчитать удельное сопротивление образца:
$$\rho=\frac{d a}{l} R_{\text{об}} (\text{Oм} \text{ cм})$$

или обратную величину - удельную электропроводность: 
$$\sigma=1 / \rho\left(\text{Oм}^{-1} \text{cм}^{-1}\right)$$

\section{Схема экспериментальной установки}
Внешний вид установки можно увидеть на рис. \ref{fig:6.1}, а её схему – на рис. \ref{fig:6.2}. 

\begin{figure}[h!]
	\centering
	\includegraphics[width = .9\linewidth]{img/ust.jpg}
	\caption{Внешний вид установки}
	\label{fig:6.1}
\end{figure}

\begin{figure}[h!]
	\centering
	\includegraphics[width = .9\linewidth]{img/scheme-2.jpg}
	\caption{Схема установки}
	\label{fig:6.2}
\end{figure}

Блок питания (1) содержит в себе два регулируемых стабилизатора тока (для образца и эталонного резистора) и регулируемый
источник питания нагревателя образца, напряжение на выходе которого контролируется вольтметром $V_\text{н}$. На верхней
крышке измерительного блока (2) находится трубчатый керамический нагреватель, в котором размещён исследуемый образец и
термопара для измерения температуры. Нагреватель с образцом и термопарой закрыты защитным цилиндром. В корпусе
измерительного блока (2) располагается эталонный резистор Rэ, переключатели направления тока образца и компенсации К1 и
К2, индикатор компенсации V с переключателем чувствительности «Точно». Измерение токов образца и компенсации
производится миллиамперметрами А1 и А2 для измерения ЭДС термопары используется милливольтметр Vт, показания которого
пересчитываются в температуру по градуировочному графику (рис. \ref{fig:6.3}).

\begin{figure}[h!]
	\centering
	\includegraphics[width = .9\linewidth]{img/grad.jpg}
	\caption{График соответствия ЭДС термопары и температуры спая}
	\label{fig:6.3}
\end{figure}


\section*{Эксперимент}
\textbf{Оборудование}
\begin{enumerate}
	\item $R{\text{э}} = 10$ Ом.
	\item Образец $l = 7$ см, $d=1.4$ см, $a = 4$ см $x = 20$ см.
\end{enumerate}

Произвели измерение электропроводности образца при комнатной температуре. Установив ток образца 5-10 мА
добились нулевого отклонения индикатора. Аналогичное сделали, сменив направление тока.

Такие же измерения провели при различных температурах. Сняли температурную зависимость тока компенсации (для двух направлений тока при каждом
значении температуры $I_{k1},I_{k2}$).

Далее взяли среднее значение тока:
$$I_k=\frac{I_{k1}+I_{k2}}{2}$$

Рассчитали проводимость для каждой снятой точки по формуле:

$$\sigma = \frac{l}{ad} \frac{I_{\text{об}}}{ I_k ~ R_{\text{э}}}$$



\subsection*{Обработка результатов измерений}
Построили график полученной зависимости $ln~\sigma(\frac{10^3}{T})$ (Т – абсолютная температура в градусах К).

\begin{figure}[h!]
	\centering
	\includegraphics[width = .8\linewidth]{graphs/lns.png}
	\caption{}
	\label{fig:exp.1}
\end{figure}
Прологарифмировали выражение \eqref{eq:4.4} и нашли связь между угловым коэффициентом наклона кривой
$ln~\sigma(\frac{10^3}{T})$ и величиной $W_g$:

$$ ln(\sigma) = \underbrace{ln(\sigma_C)}_{const} - W_g /2 k_B T $$
$$ W_g = -1000 \cdot tan(\theta) \cdot 2 k_B $$
где $tan(\theta)$ - тангенс угла наклона графика $ln~\sigma(\frac{1}{T})$, $k_B \approx 8.62 \cdot 10^{-5}$ эВ/К -
постоянная Больцмана.

Определили угловой коэффициент наклона кривой в области высоких температур и рассчитали значение $W_g$:
$$tan(\theta) \approx -3.34 $$
$$ W_g \approx 0.58 \text{ эВ}$$ 

В области истощения примесей определить зависимость $\sigma = f(T)$, считая, что $\sigma \approx T^n$. Ее
можно найти, взяв на кривой две точки и воспользовавшись соотношением $\sigma_{I} / \sigma_{2}=\left(T_{I} /
T_{2}\right)^{n}$.

????


Определили (экстраполяцией по графику) величину $\sigma_c$, соответствующую
электропроводности вещества при $T \to \infty$.
$$\sigma_c = e^7 $$
выше бред ???????????


\end{document}
